\documentclass[a4paper,12pt]{report}

% ------------------------
% Pacchetti di base
% ------------------------
\usepackage[utf8]{inputenc}
\usepackage[T1]{fontenc}
\usepackage[italian]{babel}
\usepackage{graphicx}
\usepackage{geometry}
\geometry{margin=3cm}
\usepackage{setspace}
\onehalfspacing % interlinea 1.5

% ------------------------
% Inizio documento
% ------------------------
\begin{document}

% ------------------------
% Frontespizio
% ------------------------
\begin{titlepage}
    \centering
    
    % Logo in alto
    \includegraphics[width=6cm]{tesi/logo_uniroma3.jpeg}\par\vspace{0.5cm}
    
    {\large UNIVERSITÀ DEGLI STUDI ROMA TRE}\par
    \vspace{0.2cm}
    {\normalsize Dipartimento di Ingegneria Civile, Informatica e delle Tecnologie Aeronautiche}\par
    {\normalsize Corso di Laurea Triennale in Ingegneria Informatica}\par
    \vspace{1.5cm}
    
    {\large \textbf{Tesi di Laurea Triennale}}\par
    \vspace{1.5cm}
    
    {\Large \textbf{Sfruttare la fragility di un timetable \\ 
    per l'inserimento di un treno aggiuntivo \\ 
    in caso di eventi speciali}}\par
    \vspace{2cm}
    
    % Laureanda centrata
    \begin{center}
    \textbf{Laureanda} \\
    Alessia Ragheb \\
    Matricola 550427
    \end{center}
    
    \vspace{1.2cm}
    
    % Relatrice a sinistra
    \begin{flushleft}
    \textbf{Relatrice} \\
    Prof.ssa Marcella Samà
    \end{flushleft}
    
    \vfill
    
    {\normalsize Anno Accademico 2024/2025}\par
\end{titlepage}

% ------------------------
% Indice
% ------------------------
\tableofcontents
\newpage

% ------------------------
% Capitoli (scheletro)
% ------------------------

\chapter{Introduzione}
% Qui scrivi introduzione e contesto del problema

\chapter{Stato dell'arte}
% Qui inserisci riferimenti a Wilson, Castro, altri articoli, concetto di fragilità, timetable, ecc.

\chapter{Metodologia}
% Qui spieghi il modello, l'algoritmo, le classi del codice, come funziona la DP, ecc.

\chapter{Caso di studio}
% Qui racconti il dataset (XML, JSON), le finestre artificiali/reali, i grafici Gantt, gli scenari, ecc.

\section{Prova}
\subsection{Prova2}
ciao
\chapter{Risultati}
% Qui metti i risultati numerici, fragilità trovata, esempi di percorsi ottimi, confronto scenari.

\chapter{Conclusioni e sviluppi futuri}
% Qui scrivi le conclusioni + possibili estensioni (es. cambiare timetable, scenari più complessi, ecc.)

% ------------------------
% Fine documento
% ------------------------
\end{document}
